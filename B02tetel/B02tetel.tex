\chapter{F\'azist\'er, Liouville-t\'etel és -egyenlet, s\H{u}r\H{u}s\'egm\'atrix és a Neumann-egyenlet}
 
 \section{Fázistér, Liouville-t\'etel és -egyenlet}
  
  \subsection{Liouville-tétel}
   
   Tekintsünk egy zárt rendszert és annak egy fázistérfogat tartományát ($\Gamma$). A térfogat minden pontjából elindítjuk az időfejlődést a kanonikus egyenleteknek megfelelően. A kérdés, hogy a fázistérfogat, hogyan változik az időben. 
   
   A fázissebesség a fázistér $(q,p)$ pontjában: $\vv(q,p)=\big(\{\dot{q}_i\},\{\dot{p}_i\}\big)$. Számoljuk ki ennek a divergenciáját:
   \al{
    \divo{\vv}(q,p)
     =\suml{i=1}{f}\left(\pder{\dtq_i}{q_i}+\pder{\dtp_i}{p_i}\right)
     =\suml{i=1}{f}\left(\pder{^2 H}{q_i\partial p_i}-\pder{^2 H}{p_i\partial q_i}\right)
     =0,
   }
   ami azt jelenti, hogy a fázistérfogat összenyomhatatlan folyadékként viselkedik. Ha kezdetben egy $\Gamma$ térfogatból indultunk, akkor annak térfogata nem változik, az alak változásáról azonban nem állíthatunk semmit, de általában ez hosszú idő elteltével az egész fázisteret behálózza.
   
  \subsection{Liouville-egyenlet}
   
   Tekintsünk egy sokaságot, és annak a fázisterében egy $\Gamma$ tartományt. Az ebben található fázispontok száma:
   \al{
    n_\Gamma=\intl{\Gamma}{}\dd^{dN}q\dd^{dN}p\,\rho(q,p,t).
   }
   Ennek a szubsztanciális deriváltja nulla, mert a $\Gamma$ térfogat együtt mozog a fázispontokkal és fázispont nem keletkezik/tűnik el. Ez kiírva (lásd \eqref{eq:08-szubsztancialisderivalt} egyenlet):
   \al{
    0
     &=\der{n_\Gamma}{t}
      =\der{}{t}\intl{\Gamma}{}\dd^{dN}q\dd^{dN}p\,\rho(q,p,t)
      =\intl{\Gamma}{}\dd^{dN}q\dd^{dN}p\,\Big[\partial_t\rho(q,p,t)+\partial_i\big(v_i\rho(q,p,t)\big)\Big]
       \\
     &=\intl{\Gamma}{}\dd^{dN}q\dd^{dN}p\,\Big[\partial_t\rho(q,p,t)+\divo{\vv}\rho(q,p,t)+\vv\grad\rho(q,p,t)\Big]\\
     &=\intl{\Gamma}{}\dd^{dN}q\dd^{dN}p\,\Big[\partial_t\rho(q,p,t)+\vv\grad\rho(q,p,t)\Big],
   }
   ez pedig tetszőleges $\Gamma$ térfogatra igaz, így 
   \al{
    0
     &=\partial_t\rho(q,p,t)+\vv\grad\rho(q,p,t)
      =\partial_t\rho(q,p,t)+\suml{i}{}\left(\dtq_i\pder{\rho}{q_i}+\dtp_i\pder{\rho}{p_i}\right)\\
     &=\partial_t\rho(q,p,t)+\suml{i}{}\left(\pder{H}{p_i}\pder{\rho}{q_i}-\pder{H}{q_i}\pder{\rho}{p_i}\right)
      =\partial_t\rho(q,p,t)-\{H,\rho\}\\
     \partial_t\rho(q,p,t)&=\{H,\rho\}
   }
   
   Egyensúlyi eloszlásnál a valószínűségi sűrűségnek nem lehet explicit időfüggése, így $0=\partial_t\rho(q,p,t)$, vagyis $\rho$ mozgásállandó. Emiatt csak más mozgásállandóktól függhet, azonban ezek (pl.\ $\Lv$, $\Pv$) kitranszformálhatóak, így $\rho=\rho(H(q,p))$. 
  
%   Valójában az energia nem állandó sosem $E\in[E,E+\delta E]$, így elvileg ismerni kellene $\rho(E)$ eloszlását. Ezt azonban nem ismerjük, a tapasztalat viszont azt sugallja, hogy az ekvienergiás felületen minden mikroállapot azonos valószínűségű.  
  
 \section{Sűrűségmátrix, Neumann-egyenlet}
  
  A sűrűségmátrixok definícióját, tulajdonságait lásd \aref{ss:01-mereselmelet}. fejezetben. 
  
  Itt annyi kiegészítést tennék, hogy ha a rendszer zárt, akkor az állapota reprezentálható egy Hilbert-tér elemmel, így akkor a sűrűségmátrixa a tiszta állapotnak megfelelő. Nyílt rendszernél, ahol leválasztjuk a környezetet, a rendszer kevert állapotba kerül. 
%   Tekintsünk egy zárt kvantummechanikai rendszert. Schrö\-din\-ger-kép\-ben az időfejlődés a Schrö\-din\-ger-egy\-en\-let alapján:
%   \al{
%    i\partial_t\ket{\Psi}=\opH\ket{\Psi},
%   }
%   illetve Heisenberg-képben:
%   \al{
%    \der{\opA}{t}=\frac{i}{\hbar}\big[\opH,\opA\big].
%   }
%   
%   Legyen $\ket{\phi_i}$ egy teljes ortonormált bázis. Ezen kifejtve a rendszer egy $\Psi$ állapota: $\ket{\Psi}=\suml{i}{}\ket{\phi_i}\bra{\phi_i}\et{\Psi}=\suml{i}{}c_i\ket{\phi_i}$, ahol $c_i=\bra{\phi_i}\et{\Psi}$.
%   
%   A $\ket{\Psi}$ állapothoz tartozó sűrűségoperátor: $\op{\rho}=\ket{\Psi}\bra{\Psi}$. Ennek bármelyik hatványa: $\op\rho^n=\op\rho$, hiszen $\ket{\Psi}$ normált. A sűrűségoperátor egy mátrixeleme:
%   \al{
%    \rho_{ij}
%     =\bra{\phi_i}\op{\rho}\ket{\phi_j}
%     =\bra{\phi_i}\et{\Psi}\bra{\Psi}\et{\phi_j}
%     =c_i c_j^*
%   }
%   Az előző összefüggésből látszik, hogy $\tr{\op\rho}=\tr{\op\rho^2}=1$.
%   
%   Egy tetszőleges operátor kvantummechanikai várható értéke ebben az állapotban:
%   \al{
%    \mv{A}
%     =\bra{\Psi}\opA\ket{\Psi}
%     =\tr\big(\bra{\Psi}\opA\ket{\Psi}\big)
%     =\tr\big(\ket{\Psi}\bra{\Psi}\opA\big)
%     =\tr\big(\op\rho\opA\big).
%   }
%   
%   Koordináta-reprezentációban $\rho(x,x')=\Psi(x)\Psi^*(x')$, melynek diagonális része: $\rho(x,x)=\Psi(x)\Psi^*(x)=\abs{\Psi}^2$, ami a megtalálási valószínűség-sűrűséggel egyenlő. 
%   
%   Tekintsünk most egy nyílt rendszert. Legyen ez egy nagy rendszer alrendszere, az ezen kívüli rész pedig a környezet. A teljes rendszer Hamilton-operátora: $\opH=\opH_\text{R}+\opH_\text{K}$. Mivel a két alrendszer kölcsönhat, ezért a rendszer állapotát nem lehet csak $\opH_\text{R}$ sajátfüggvényeivel leírni. A sűrűségoperátor azonban definiálható:
%   \al{
%    \op\rho_\text{R}=\tr_\text{K}\rho,
%   }
%   ahol $\rho$ a teljes rendszer $\Psi$ állapothoz tartozó hullámfüggvénnyel elkészített sűrűségoperátor és $\tr_\text{K}$ egy parciális trace művelet, ami a a környezet szabadsági fokaira végzi csak el a nyomképzést. Praktikusan ez azt jelenti, hogy szendvicsel a $\opH_\text{K}$ teljes sajátfüggvényrendszerével. 
%   
%   Ha a környezettel való kapcsolat állandó, $\rho_\text{R}$ kifejthető a $\opH_\text{R}$ teljes sajátfüggvényrendszeréből képzett projektorok bázisán:
%   \al{
%    \rho_\text{R}=\suml{i}{}P_i\ket{\Psi_{\text{R},i}}\bra{\Psi_{\text{R},i}}.
%   }
  Vizsgáljuk meg, hogy milyen a sűrűségoperátornak időfejlődése. Ehhez:
  \al{
   \pder{}{t}\mv{\opA}
    &=\pder{}{t}\tr\left(\op\rho\opA\right)
     =\tr\left(\pder{\op\rho}{t}\opA\right)\\
    &=\pder{}{t}\suml{i=1}{N}p_i\bra{\psi_i}\hat{A}\ket{\psi_i}
     =\suml{i=1}{N}p_i\left(\pder{}{t}\bra{\psi_i}\hat{A}\ket{\psi_i}+\bra{\psi_i}\hat{A}\pder{}{t}\ket{\psi_i}\right)\\
    &=\suml{i=1}{N}p_i\left(\frac{1}{-i\hbar}\bra{\psi_i}\opH\hat{A}\ket{\psi_i}+\bra{\psi_i}\hat{A}\frac{1}{i\hbar}\opH\ket{\psi_i}\right)
     =\suml{i=1}{N}p_i\bra{\psi_i}\big[\opH,\hat{A}\big]\ket{\psi_i}\\
    &=\tr\left(\op\rho\frac{i}{\hbar}\big[\opH,\hat{A}\big]\right)
     =\tr\left(\frac{i}{\hbar}\big[\op\rho,\opH\big]\hat{A}\right),
  }
  ahol kihasználtuk, hogy a $\tr$ alatt az operátorok ciklikusan felcserélhetőek. Ennek minden $\opA$-ra igaznak kell lennie, így:
  \al{
   \pder{\op\rho}{t}=\frac{i}{\hbar}\big[\op\rho,\opH\big],
  }
  ami a Neumann-egyenlet. Vegyük észre, hogy ez az egyenlet éppen a Liouville-egyenlet kvantummechanikai megfelelője, abból a $\{\cdot,\cdot\}\to\frac{1}{i\hbar}[\cdot,\cdot]$ analógiával kapható. Ez egy előjelben különbözik a mérhető mennyiségekre levezetett Heisenberg-képbeli mozgásegyenlettől, azonban ez nem ellentmondás, mert a sűrűségoperátor egyrészt nem mérhető mennyiség, másrészt pedig az egyenletet a Schrödinger-képben vezettük le.
  
  Egyensúlyban, mikor a rendszer időfüggetlen állapotban van (a bal oldal nulla), így a sűrűségoperátor kommutál a Hamilton-operátorral, vagyis $\op\rho=\text{állandó}$, itt is csak az energiától függ. Mivel a két operátor kommutál, ezért létezik közös sajátfüggvény-rendszerük. Ekkor $\op\rho$ diagonalizálható, a diagonálisban $p_i$ valószínűségek állnak, melyek megadják, hogy a rendszer az $i$-edik energia sajátállapotban mekkora valószínűséggel tartózkodik.
  
  
